\documentclass[]{article}
\usepackage{lmodern}
\usepackage{amssymb,amsmath}
\usepackage{ifxetex,ifluatex}
\usepackage{fixltx2e} % provides \textsubscript
\ifnum 0\ifxetex 1\fi\ifluatex 1\fi=0 % if pdftex
  \usepackage[T1]{fontenc}
  \usepackage[utf8]{inputenc}
\else % if luatex or xelatex
  \ifxetex
    \usepackage{mathspec}
  \else
    \usepackage{fontspec}
  \fi
  \defaultfontfeatures{Ligatures=TeX,Scale=MatchLowercase}
\fi
% use upquote if available, for straight quotes in verbatim environments
\IfFileExists{upquote.sty}{\usepackage{upquote}}{}
% use microtype if available
\IfFileExists{microtype.sty}{%
\usepackage{microtype}
\UseMicrotypeSet[protrusion]{basicmath} % disable protrusion for tt fonts
}{}
\usepackage[margin=1in]{geometry}
\usepackage{hyperref}
\hypersetup{unicode=true,
            pdftitle={Final Report for Red Wine Analysis},
            pdfauthor={Ruiqiang Chen, Michael DeWitt, David Williams, Alex Vannoy},
            pdfborder={0 0 0},
            breaklinks=true}
\urlstyle{same}  % don't use monospace font for urls
\usepackage{longtable,booktabs}
\usepackage{graphicx,grffile}
\makeatletter
\def\maxwidth{\ifdim\Gin@nat@width>\linewidth\linewidth\else\Gin@nat@width\fi}
\def\maxheight{\ifdim\Gin@nat@height>\textheight\textheight\else\Gin@nat@height\fi}
\makeatother
% Scale images if necessary, so that they will not overflow the page
% margins by default, and it is still possible to overwrite the defaults
% using explicit options in \includegraphics[width, height, ...]{}
\setkeys{Gin}{width=\maxwidth,height=\maxheight,keepaspectratio}
\IfFileExists{parskip.sty}{%
\usepackage{parskip}
}{% else
\setlength{\parindent}{0pt}
\setlength{\parskip}{6pt plus 2pt minus 1pt}
}
\setlength{\emergencystretch}{3em}  % prevent overfull lines
\providecommand{\tightlist}{%
  \setlength{\itemsep}{0pt}\setlength{\parskip}{0pt}}
\setcounter{secnumdepth}{5}
% Redefines (sub)paragraphs to behave more like sections
\ifx\paragraph\undefined\else
\let\oldparagraph\paragraph
\renewcommand{\paragraph}[1]{\oldparagraph{#1}\mbox{}}
\fi
\ifx\subparagraph\undefined\else
\let\oldsubparagraph\subparagraph
\renewcommand{\subparagraph}[1]{\oldsubparagraph{#1}\mbox{}}
\fi

%%% Use protect on footnotes to avoid problems with footnotes in titles
\let\rmarkdownfootnote\footnote%
\def\footnote{\protect\rmarkdownfootnote}

%%% Change title format to be more compact
\usepackage{titling}

% Create subtitle command for use in maketitle
\newcommand{\subtitle}[1]{
  \posttitle{
    \begin{center}\large#1\end{center}
    }
}

\setlength{\droptitle}{-2em}
  \title{Final Report for Red Wine Analysis}
  \pretitle{\vspace{\droptitle}\centering\huge}
  \posttitle{\par}
  \author{Ruiqiang Chen, Michael DeWitt, David Williams, Alex Vannoy}
  \preauthor{\centering\large\emph}
  \postauthor{\par}
  \predate{\centering\large\emph}
  \postdate{\par}
  \date{7/28/2017}

\usepackage{amsmath}
\usepackage{float}

\begin{document}
\maketitle

\section{Introduction}\label{introduction}

The purpose of this document is to report the proposed statistical
models for classification of red wine bases on 11 predictors. The
purpose of this analysis is to provide a model to the vintners in order
for them to better predict the quality rating for their product.
Analysis will be performing using both regression techniques and
classification techniques.

\section{Description of Data}\label{description-of-data}

The data set provided is the Wine dataset from UC Irvine of red
\emph{vinho verde} wine samples, from the north of Portugal {[}Cortez et
al., 2009{]}. It consists of 1599 with a total of 11 physicochemical
predictors and a response variable. These predictors include the
following: fixed\_acidity, volatile\_acidity, citric\_acid,
residual\_sugar, chlorides, free\_sulfur\_dioxide,
total\_sulfur\_dioxide, density, pH, sulphates, alcohol, quality with
the quality feature being associated with the judgement of the
individual wine's quality. Quality is the feature of interest for the
dataset as the vintner is interested in judging the wine's quality
through objective means rather than today's subjective method of
averaging the 1-10 point judgment of taste-testers. A summary of these
measures as well as the response variable can be seen in Table 1.

\begin{longtable}[]{@{}lrrrr@{}}
\caption{Summary Statistics for the Wine Dataset}\tabularnewline
\toprule
Descriptions & min & median & mean & max\tabularnewline
\midrule
\endfirsthead
\toprule
Descriptions & min & median & mean & max\tabularnewline
\midrule
\endhead
fixed acidity (g(tartaric acid)/dm\^{}3) & 4.60 & 7.90 & 8.32 &
15.90\tabularnewline
volatile acidity (g(acetic acid)/dm\^{}3) & 0.12 & 0.52 & 0.53 &
1.58\tabularnewline
citric acid (g/dm\^{}3) & 0.00 & 0.26 & 0.27 & 1.00\tabularnewline
residual sugar (g/dm\^{}3) & 0.90 & 2.20 & 2.54 & 15.50\tabularnewline
chlorides (g(sodium cloride)/dm\^{}3 & 0.01 & 0.08 & 0.09 &
0.61\tabularnewline
free sulfur dioxide (mg/dm\^{}3) & 1.00 & 14.00 & 15.87 &
72.00\tabularnewline
total sulfur dioxide (mg/dm\^{}3) & 6.00 & 38.00 & 46.47 &
289.00\tabularnewline
density (g/cm\^{}3) & 0.99 & 1.00 & 1.00 & 1.00\tabularnewline
pH & 2.74 & 3.31 & 3.31 & 4.01\tabularnewline
sulphates (g(potassium sulphate)/dm3) & 0.33 & 0.62 & 0.66 &
2.00\tabularnewline
alcohol (\% vol.) & 8.40 & 10.20 & 10.42 & 14.90\tabularnewline
quality & 3.00 & 6.00 & 5.64 & 8.00\tabularnewline
\bottomrule
\end{longtable}

The distribution of these different criteria can be seen below in the
histograms in Figure 1.

\begin{figure}[htbp]
\centering
\includegraphics{Final_Report_Write_Up_files/figure-latex/global_histograms-1.pdf}
\caption{Histogram of all variables in the data set}
\end{figure}

The following are slightly right skewed: Fixed Acidity, Volatile
Acidity, Citric Acid, , Free Sulfur Dioxide, Total Sulfur Dioxide,
Sulphates, and Alcohol. Residual Sugar and Chlorides are heavily right
skewed with density and pH appearing more normally distributed.
Completing a Shapiro Wilke normality test on the components indicates
that all are non-normal. Reviewing the individual components there
appears to be a slight irregularity with total free sulfur dioxide. This
can be seen in the histogram of this variable.

\begin{figure}[htbp]
\centering
\includegraphics{Final_Report_Write_Up_files/figure-latex/so2_fig-1.pdf}
\caption{Histogram of Sulfur Dioxide Predictor}
\end{figure}

As well as the fit of free sulfur dioxide display high studentized
residuals and leverage and thus should be considered for removal in the
modeling process. These wines are 1080 and 1082.

\begin{figure}[htbp]
\centering
\includegraphics{Final_Report_Write_Up_files/figure-latex/so2_resid-1.pdf}
\caption{Residual Plots of Linear Model Predicting Quality with Total
Sulfur Dioxide}
\end{figure}

These two wines have been removed from the clean dataset in order to be
better predictors. The presence of these two wines may result in
incorrect or inaccurate predictions. Ass we did not gather this dataset,
we do not know if this information was incorrectly captured or if these
values are real. However, given the strong indication that these two
points are outliers with high leverage it is a good assumption to remove
these two points.

\section{Method}\label{method}

In order to estimate the test error of any of the generated models, the
data was divided in testing and training data sets with which to train
then models and then test and estimate the testing error. Seventy
percent of the raw data was randomly selected and placed in the training
set with the remaining thirty percent used as the testing data set.

\subsection{Regression}\label{regression}

In order to select the best fit regression model, several different
modeling methods were produced. These include Least Squares Regression,
Ridge Regression, Lasso Regression, Principle Components Regression, and
Partial Least Squares Regression. The quality integer was the value that
the model attempts to predict for each of these methods. The data was
divided into two sets, a training set to train the model and a testing
set for model validation. We will now go deeper in the model generation
process for each of these different modeling types and methods.

\subsubsection{Least Squares}\label{least-squares}

The least squares regression method that was tested was the best subset
selection. The methodology used to determine the best subset model was
to first run cross-validation on the training set in order to determine
the best number of predictors to include in the model. This analysis
indicated that any additional predictor after three variables were
selected did not increase the accuracy of the model. The training data
was then used to determine the best subset of the linear model with
three predictors. The best subset included:

\paragraph{Residual Analysis}\label{residual-analysis}

Here we need to make some plots against of the fit vs predictors and fit
vs prediction to cross off that we considered our residuals

\begin{figure}[htbp]
\centering
\includegraphics{Final_Report_Write_Up_files/figure-latex/residuals-1.pdf}
\caption{Plot of Residuals from Linear Regression}
\end{figure}

The residuals appear to have no distinct pattern which is a positive
sign that there are not lurking relationships that have not been treated
by the modeling.

\subsubsection{Ridge Regression}\label{ridge-regression}

Ridge regression was performed on the dataset as well. Cross-validation
was performed on the training data set to determine the optimum value
for lambda for the ridge regression. This lambda, 0.079 was then used in
a ridge regression model with the testing dataset. Further, there seems
to be a very large coefficient with 11 much smaller coefficients.

\subsubsection{Lasso Regression}\label{lasso-regression}

Lasso regression was used with cross-validation on the data set.
Cross-validation was used to determine the best lambda which was 0.005.
As a function of the lasso regression only pH was shrunk to zero with
total sulphur dioxide and free sulphur dioxide being shrunk to near
zero. Also similar to ridge regression, there appears to be one much
larger coefficient in relation to the others.

\begin{figure}[H]

{\centering \includegraphics[width=225px]{Final_Report_Write_Up_files/figure-latex/lasso-1} \includegraphics[width=225px]{Final_Report_Write_Up_files/figure-latex/lasso-2} 

}

\caption{Plot of Lambda vs Coefficients for Ridge/Lasso Regression}\label{fig:lasso}
\end{figure}

\subsubsection{Principal Components
Regression}\label{principal-components-regression}

Principal components regression was used. Based on the analysis of the
principal components, the first nine principal components were used to
be trained on the training set. This was done because 90\% of the
variation could be explained by these first nine components. This is
graphically displayed in the plot of principal components.

\begin{figure}[htbp]
\centering
\includegraphics{Final_Report_Write_Up_files/figure-latex/pcr_fig-1.pdf}
\caption{Variance Explained for Principal Components Regression}
\end{figure}

\subsubsection{Partial Least Squares
Regression}\label{partial-least-squares-regression}

Partial least squares regression was used. However, the difference is
that it uses quality response as supervision over the principal
components. Using this method one can see from the plot of patial least
squares components that after roughly 2-3 components, the model accuracy
does not increase drastically.

\begin{figure}[htbp]
\centering
\includegraphics{Final_Report_Write_Up_files/figure-latex/pls_fig-1.pdf}
\caption{Variance Explained for Partial Least Squares Regression}
\end{figure}

\subsubsection{Boosted Regression}\label{boosted-regression}

Boosted regression was also used. The interaction depth was limited to
four in order to reduce the likelihood of over-fitting the data. The
model was trained on 5,000 different trees.

\begin{figure}[H]
\centering
\includegraphics{Final_Report_Write_Up_files/figure-latex/boost_importance-1.pdf}
\caption{Relative Importance from Boosted Regression}
\end{figure}

\subsubsection{Model Selection}\label{model-selection}

The resulting mean squared errors for each regression method were
tabulated in order to determine the superior model.

\begin{figure}[H]
\centering
\includegraphics{Final_Report_Write_Up_files/figure-latex/mse_results-1.pdf}
\caption{Plot of Results of Different Regression Techniques}
\end{figure}

\subsection{Classification}\label{classification}

For classification purposes the wines were segregated in to three
different classes. These classes include ``good'' (\(quality >7\)),
``medium'' (\(\; quality \; between \; 4 \;and\; 7\)) and
``poor''(\(quality < 4\)).

\subsubsection{Model Selection}\label{model-selection-1}

Several different classification models were used in this analysis given
the new variable added to the data set. The methods used were K-Nearest
Neighbors, Linear Discriminant Analysis, Quadratic Discriminant
Analysis, and tree classification. These different models were trained
on the training data set and then applied to the testing dataset to
estimate the accuracy. It is important to note that greater accuracy was
achieved by scaling values for the K-Nearest Neighbors approach as this
approach uses Euclidean distances and thus is sensitive to scale
differences. The phenomena can be seen as with the unscaled values the
validation algorithm found that 17 were used versus 64. The larger
number of neighbours makes for a much more global model, less sensitive
to immediate neighbours in the bias versus variance trade off. The tree
classification model was trained first through cross-validation and then
pruned to six leaves in order to reduce the impact of over-fitting in
the bias variance trade off.

\subsection{Comparison of Models}\label{comparison-of-models}

\begin{figure}[htbp]
\centering
\includegraphics{Final_Report_Write_Up_files/figure-latex/classification_results-1.pdf}
\caption{Plot of Results of Different Classification Techniques}
\end{figure}

All of these models seek to maximize the global accuracy of the model.
More interesting for the vintners is the ability to detect each of the
three different classes of the wines.

\begin{longtable}[]{@{}lrrr@{}}
\caption{Detailed Classification Accuracy}\tabularnewline
\toprule
Method & Good & Medium & Poor\tabularnewline
\midrule
\endfirsthead
\toprule
Method & Good & Medium & Poor\tabularnewline
\midrule
\endhead
KNN & 0.24 & 0.98 & 0.00\tabularnewline
Naive Bayes & 0.52 & 0.77 & 0.06\tabularnewline
LDA & 0.43 & 0.93 & 0.06\tabularnewline
QDA & 0.61 & 0.90 & 0.89\tabularnewline
Tree & 0.36 & 0.92 & 0.00\tabularnewline
\bottomrule
\end{longtable}

\section{Discussion}\label{discussion}

stuff to write about \pagebreak

\section{Conclusion}\label{conclusion}

This analysis shows that for regression the boosted regression resulted
in the highest accuracy of all regression models; however, this accuracy
comes at a cost of interpretability. Because the boosted algorithms have
little intrepreation this accuracy is more for prediction than
inference. If inference is the goal for the vintner and
horticulturalists who seek to understand the properties that make good
wines, the model with higher interpretability and the second highest
accuracy is the Lasso regression model. While the PLS is more accurate,
again it suffers from ease of interpretation. Thus with this in mind,
the superior model for inference with high accuracy is characterized by
the below equation:\linebreak

\begin{equation}
\begin{aligned}
\label{lasso_eq}
quality = 39.37 + 0.0823 * fixed\;acidity -0.981 * volatile\;acidity -0.405 * citric\;acid \\
-0.013 *residual\;sugar -1.075 * chlorides + 0.006 * free\;sulfur\;dioxide\\
- 0.002 * total\;sulfur\;dioxide - 37.09 * density +1.032 * sulphates + 0.256 * alcohol
\end{aligned}
\end{equation}

Thus from equation \ref{lasso_eq} the vintner can examine each of the
variables independently and provide some degree of inference regarding
the chemical levels that influence the quality of the red wine. For
instances one can see that sulfate content appears to have a stronger
positive influence on the wine quality while wines with additional
residual sugars reduce the quality score. In the hands of the vinter,
these relationships can be explored or potentially exploited to produce
a higher quality wine more consistently. \linebreak

Turning to the classification method, the best overall classification
method was Quadratic Discriminate Analysis. This is seen in both the
overall accuracy as well in its ability to accurately classify each
subcategory. While the other methods have lesser abilities to detect the
good and poor quality wines, the QDA method showed the best accuracy in
these two fields, which is very important for vintners when it comes to
pricing and selling. The penalty of misclassifying a good wine as medium
or a poor wine as medium/ good is severe as this may damage the
reputation of the winery. From this analysis it is clear that QDA is the
superior method for classification of red wines given this dataset.

\section{Discussion of Issues}\label{discussion-of-issues}

Outlier filtering Colinnearity (some points became clear were related
fixed acidity, citric acid, pH. Some methods did better filtering this
impact PCR, PLS, Lasso, Ridge, while linear regression suffered a
little, but we removed pH as a predictor and that helped.) Non-normality
-\textgreater{} we could have done some advanced transformations like
boxcox, but this would improve model accuracy at the expense of
inference.


\end{document}
